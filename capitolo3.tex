\label{capitolo3}
\section{Teoria della stima puntuale}
\subsection{Definizioni}
Sia X una va con funzione di ripartizione F e densit� di probabilit� f non completamente specificata, ovvero con un parametro $\theta$ m-dimensionalea valori in $\Theta$ sottoinsieme di $\mathbb{R}^m$.
Definiamo la statistica T come una variabile aleatoria funzione del campione $T=g(X_1, \dots ,X_n)$. La distribuzione di una statistica T � detta \emph{ distribuzione (o legge) campionaria}.
Una statistica T non dipende mai dal parametro incognito mentre la distribuzione campionaria in generale dipender� da $\theta$.\\
Una funzione $\kappa: \Theta \rightarrow \mathbb{R}$ � detta \emph{caratteristica della popolazione}.\\
Sianno $X_1, \dots ,X_n \sim f(x,\theta), \; \theta \in \Theta$ e $\kappa(\theta)$ una caratteristica della popolazione. Uno stimatore di $\kappa(\theta)$ basato sul campione $X_1, \dots ,X_n$ � una statistica $T=g(X_1, \dots ,X_n)$ usata per stimare $\kappa(\theta)$. Il valore assunto dallo stimatore � detta stima di $\kappa(\theta)$.
\subsubsection{Errore quadratico medio}
Potremmo trovarci, in un problema di stima, nella situazione di dover decidere fra stimatori diversi della stessa caratteristica $\kappa(\theta)$. Utilizzeremo come criterio di scelta la media della prossimit� di T a $\kappa(\theta)$
che esprimiamo in termini di $E_\theta[(T-\kappa(\theta))^2]$.\\
Se T � stimatore di $\kappa(\theta)$ tale che $E_\theta[(T-\kappa(\theta))^2]<\infty \ \forall \theta \in \Theta$, allora $E_\theta[(T-\kappa(\theta))^2]$ � detto \emph{errore quadratico medio} di T rispetto a $\kappa(\theta)$.\\
Il MSE di uno stimatore T esiste solo se T ha media e varianza finite, o, equivalentemente, se e solo se ha momento secondo finito.\\
\textbf{Osservazione:} Per calcolare l'errore quadratico medio � utile decomporlo nel seguente modo:
$$E_\theta[(T-\kappa(\theta))^2]= Var_\theta(T)+[E_\theta(T)-\kappa(\theta)]^2$$ 
in cui la quantit� $[E_\theta(T)-\kappa(\theta)]$ � detta \emph{distorsione di bias}.\\
Tra tutti gli stimatori di $\kappa(\theta)$ ci piacerebbe scegliere quello con MSE minore. Questa cosa equivale a minimizzare contemporaneamente varianza e distorsione;  ma questo � impossibile per ogni $\theta$ perci� ci accontentiamo di utilizzare la sottoclasse di stimatori che hanno distorsione nulla; questa classe � detta di stimatori \emph{non distorti o corretti}. L'erroe quadratico medio di uno stimatore non distorto coincide con la sua varianza.
\subsection{Stimatori non distorti}
Una statistica T che ammette media per ogni $\theta$ in $\Theta$ � detta \emph{stiamtore non distorto} della caratteristica $\kappa(\theta)$ se 
$$E_\theta(T)=\kappa(\theta) \qquad \forall\theta\in\Theta$$
\\
Se $X_1,\dots,X_n iid \sim f(x,\theta), \ \theta\in\Theta$ ed $E_\theta(X_1)$ esiste qualunque sia $\theta$ allora $E_\theta(\overline{X})=E_\theta(X_1), \ \forall\theta$ e quindi:
\begin{center}
\emph{La media campionaria $\overline{X}$ � stimatore non distorto della media teorica}.\\
\emph{La varianza campionaria $S^2$ � stimatore non distorto della varianza teorica.}
\end{center}
\subsection{Propriet� asintotiche degli stimatori}
\begin{defn}
Sia $X_1,\dots,X_n$ una successione di variabili aleatorie i.i.d. con comune funzione di densit� $f(x,\theta), \ \theta \in \Theta $ e sia $T_n$ uno stimatore di $\kappa(\theta)$ che � funzione delle prime n osservazioni. La successione $\{T_n\}_n$ � \emph{asintoticamente non distorta} per $\kappa(\theta)$ se
$$\lim_{n \to \infty}E_\theta(T_n)=\kappa(\theta) \qquad \forall\theta \in \Theta$$
\end{defn}
%\\
\begin{defn}
Sia $X_1,\dots,X_n$ una successione di variabili aleatorie i.i.d. con comune funzione di densit� $f(x,\theta), \ \theta \in \Theta $ e sia $T_n$ uno stimatore di $\kappa(\theta)$ che � funzione delle prime n osservazioni. La successione $\{T_n\}_n$ � \emph{consistente in media quadratica} per $\kappa(\theta)$ se
$$\lim_{n \to \infty}E[(T_n-\kappa(\theta))^2]=0 \qquad \forall\theta \in \Theta$$
\end{defn}
%\\
\begin{defn}
Sia $X_1,\dots,X_n$ una successione di variabili aleatorie i.i.d. con comune funzione di densit� $f(x,\theta), \ \theta \in \Theta $ e sia $T_n$ una statistica funzione soltanto delle prime n osservazioni. La successione $\{T_n\}_n$ � \emph{asintoticamente gaussiana} con \emph{media asintotica} $\mu_n(\theta)$ e \emph{varianza asintotica} $\sigma^2_n(\theta)$ se
$$\lim_{n \to \infty}P\Bigg(\frac{T_n-\mu_n(\theta)}{\sigma_n(\theta)}\leq z\Bigg)= \Phi(z), \qquad \forall z \in \mathbb{R}$$
\end{defn}
\subsection{Diseguaglianza di Fr�chet-Cramer-Rao}
Abbiamo visto nei capitoli precedenti come per uno stimatore non distorto ridurre il MSE significhi ridurre la varianza dello stimatore. Ora ci chiediamo qual'� la minima varianza che lo stimatore pu� avere e quale sia lo stimatore con tale varianza.\\
Siano $X_1,\dots,X_n$ variabile aleatorie iid con comune densit� $f(x,\theta), \ \theta \in\Theta$ e sia $T=g(X_1,\dots,X_n)$ uno stimatore non distorto della caratteristica $\kappa(\theta)$ a varianza finita. Assumiamo che le seguenti condizioni di regolarit� siano soddisfatte:
\begin{enumerate}
\item $\Theta$ � un intervallo aperto in $\mathbb{R}$;
\item $S=\{x: \ f(x,\theta)>0$ � indipendente da $\theta$;
\item $\theta \mapsto f(x,\theta)$
 � derivabile su $\Theta, \; \forall x\in S$;
\item $E_\theta\Bigg(\frac{\partial}{\partial \theta}log f(X_1,\theta)\Bigg)=0 \quad \forall\theta\in\Theta$;
\item $0<E_\theta\Bigg[\Bigg(\frac{\partial}{\partial \theta}log f(X_1,\theta)\Bigg)^2\Bigg]<\infty \quad \forall\theta\in\Theta$;
\item $\kappa: \Theta \rightarrow \mathbb{R}$ � derivabile su $\Theta$ e:
$$k'(\theta)=E_\theta\bigg(T\frac{\partial}{\partial\theta}log L_\theta(X_1,\dots,X_n)\Bigg)\qquad \forall\theta\in\Theta$$
\end{enumerate}

Allora
\begin{equation}
Var_\theta(T)\geq \frac{(\kappa'(\theta))^2}{nI(\theta)} \quad \forall\theta\in\Theta
\end{equation}

Dove $$I(\theta)=E_\theta\Bigg[\Bigg(\frac{\partial}{\partial \theta}log f(X_1,\theta)\Bigg)^2\Bigg]$$
� nota come informazione di Fisher.\\
\\
\textbf{Dimostrazione} Per maggiore semplicit� introduciamo le variabili aleatorie $Y_1,\dots,Y_n$ definite da 
$$Y_j= \frac{\partial}{\partial\theta}log f(X_j,\theta),\qquad\forall j=1,\dots,n$$
$Y_1,\dots,Y_n$ sono variabili aleatorie iid a media nulla e varianza finita $I(\theta) \; \forall\theta$. Infatti 
$$E_\theta(Y_j=)=E_\theta\Bigg(\frac{\partial}{\partial\theta}log f(X_j,\theta)\Bigg)=0 \qquad\quad [per \; l'ipotesi \; (4)]$$
e
$$Var_\theta(Y_j)=E_\theta(Y_j^2)=E_\theta\Bigg[\Bigg(\frac{\partial}{\partial\theta}log f(X_j,\theta)\Bigg)^2\Bigg]=I(\theta) \in (0,\infty)\qquad[per \; l'ipotesi \; (5)]$$
Inoltre
$$\frac{\partial}{\partial\theta}logL_\theta(X_1,\dots,X_n)=\frac{\partial}{\partial\theta}log \prod^n_{j=1}f(X_j,\theta)=\sum^n_{j=1}\frac{\partial}{\partial\theta}log f(X_j,\theta)=\sum^n_{j=1}Y_j$$
da cui ricaviamo che
\begin{equation}\label{eq1}
E_\theta\bigg(\frac{\partial}{\partial\theta}logL_\theta(X_1,\dots,X_n)\bigg)=\sum^n_{j=1}E_\theta(Y_j)
\end{equation}
e
\begin{equation}\label{eq2}
Var_\theta\bigg(\frac{\partial}{\partial\theta}logL_\theta(X_1,\dots,X_n)\bigg)=\sum^n_{j=1}Var_\theta(Y_j)=nI(\theta)
\end{equation}
Dall'ipotesi (6 )e dall'equazione \ref{eq1} ricaviamo che
$$\kappa'(\theta)=E_\theta \Bigg(T\frac{\partial}{\partial\theta}logL_\theta(X_1,\dots,X_n)\Bigg)=Cov\bigg(T,\frac{\partial}{\partial\theta}logL_\theta(X_1,\dots,X_n)\bigg)$$
cosicch�
$$(\kappa'(\theta))^2=\bigg[Cov\bigg(T,\frac{\partial}{\partial\theta}logL_\theta(X_1,\dots,X_n)\bigg)\bigg]^2$$
da cui per le propriet� della covarianza
$$\leq Var_\theta (T) Var_\theta\bigg(\frac{\partial}{\partial\theta}logL_\theta(X_1,\dots,X_n)\Bigg)$$
$$=Var_\theta(T) n I(\theta)\qquad [per \; l'equazione \; \ref{eq2}]$$
