\label{capitolo1}
\section{Media e varianza campionaria}
\subsection{Media campionaria}
Definiamo \textit{campione casuale} lungo \emph{n} un insieme di \textit{n} variabili aleatorie i.i.d. estratte da una popolazione di densit� \emph{f}. La media $\mu=E(X_1)$ uguale per tutte le $X_i$ del campione prende il nome di \emph{media della popolazione} e la comune varianza $\sigma^2$ � detta \emph{varianza della popolazione}.\\
Si chiama \emph{media campionaria} di un campione casuale $X_1 \dots X_n$ la quantit�:
$$\frac{X_1+ \dots +X_n}{n}$$
e si indica con  $\overline{X}_n$. La media campionaria dipende dal campione � perci� una variabile aleatoria.\\
Valore atteso e varianzaq della media campionaria possono essere facilmente calcolate tramite le propriet� di valore atteso e varianza:
$$E(\overline{X}_n)=E\Bigg(\frac{\sum^n_{j=1}{X_j}}{n}\Bigg)=\frac{E\Big(\sum^n_{j=1}{X_j}\Big)}{n}=\frac{n\times \mu}{n}= \mu$$
$$Var(\overline{X}_n)=Var\Bigg(\frac{\sum^n_{j=1}{X_j}}{n}\Bigg)= \frac{nVar(X_1)}{n^2}=\frac{\sigma^2}{n}$$
\subsection{Varianza campionaria}
Dato un campione casuale $X_1 \dots X_n$ si definisce \emph{varianza campionaria} la quantit�:
$$S^2=\frac{1}{n-1}\sum_{j=1}^{n}(X_j-\overline{X})^2$$
per dimostrare la non distorsione di questo stimatore dobbiamo innanzitutto dimostrare le seguenti espressioni:
\begin{equation}\label{f1}
\sum_{j=1}^{n}(X_j-\overline{X})^2=\sum_{j=1}^{n}(X_j-\mu)^2-n(\overline{X}-\mu)^2
\end{equation}
\begin{equation}\label{f2}
E\bigg(\sum_{j=1}^{n}(X_j-\overline{X})^2\Bigg)=(n-1)\sigma^2
\end{equation}
\textbf{Dimostrazione \ref{f1}}

$$\sum_{j=1}^{n}(X_j-\overline{X})^2=\sum_{j=1}^{n}[(X_j-\mu)-(\overline{X}-\mu)]^2$$
$$=\sum_{j=1}^{n}[(X_j-\mu)^2+(\overline{X}-\mu)^2-2(X_j-\mu)(\overline{X}-\mu)]$$
$$=\sum_{j=1}^{n}(X_j-\mu)^2+n(\overline{X}-\mu)^2-2(X_j-\mu)\sum_{j=1}^{n}(\overline{X}-\mu)$$
$$=\sum_{j=1}^{n}(X_j-\mu)^2+n(\overline{X}-\mu)^2-2n(X_j-\mu)^2$$
$$=\sum_{j=1}^{n}(X_j-\mu)^2-n(\overline{X}-\mu)^2$$

\textbf{Dimostrazione \ref{f2}}
$$E\Bigg(\sum_{j=1}^n(X_j-\overline{X})^2\Bigg)=E\sum_{j=1}^{n}(X_j-\mu)^2-nE(\overline{X}-\mu)^2$$
$$\sum_{j=1}^{n}E(X_j-\mu)^2-nVar(\overline{X})=n\sigma^2-\sigma^2$$

Da queste due formule ricaviamo che $E(S^2)=\sigma^2$ condizione necessaria e sufficente per la non distorsione dello stimatore.
\subsection{Legge dei grandi numeri}
Sia $X_1,X_2,\dots$ una successione di variabili aleatorie i.i.d. con media $\mu$ e varianza $\sigma^2$ finite e sia $\overline{X}_n$ la media campionaria, allora per ogni $\epsilon>0$:
$$\lim_{n \to \infty}P(|\overline{X}_n-\mu|>\epsilon)=0$$
Dalla diseguaglianza di Chebychev che afferma:
$$P(|X-E(X)|>\epsilon)\leq \frac{Var(X)}{\epsilon^2}$$
\subsection{Distribuzioni campionarie}
La prima densit� che introduciamo � la densit� $\Gamma(\alpha,\beta)$ questa densit� � utile in quanto permette di ricavare altre densit� favose da essa.
La sua funzione di densit� ha la forma seguente:
$$f(x,\alpha,\beta)= \frac{(1\slash \beta)^\alpha}{\Gamma(\alpha)}e^{-\frac{x}{\beta}}x^{\alpha-1} \textbf{1}_{(0,+\infty)}(x)$$
Mentre la sua funzione generatrice dei momenti �:
$$M(t)=E(e^{tX})=\frac{1}{(1-\beta t)^\alpha} \qquad \forall t < 1/\beta$$
Nel caso in cui $\alpha=1$ abbiamo che la densit� \emph{Gamma} diventa un'esponenziale di parametro beta $\xi(\beta)$.\\
 Nel caso invece in qui sia $\alpha=\frac{n}{2}$ la densit� che troviamo � una $\chi^2_n$.\\
Se $X\sim N(0,1)$ allora $X^2 \sim \chi^2_1$; se prendiamo un campione casuale $X_1, \dots ,X_n \sim N(0,1)$ allora $\sum^n_{j=1} X^2_j \sim \chi^2_n$
\\
Vediamo ora alcune propriet� che portano alle affermazioni appena fatte
\begin{enumerate}
\item Se $X \sim \Gamma(\alpha ,\beta), \ c>0$ e $Y=cX$ allora $Y \sim \Gamma(\alpha ,c \beta)$;
\item Se X,Y sono v.a. indipendenti con $X \sim \Gamma(\alpha ,\beta)$ e $Y \sim \Gamma(c,\beta)$ allora $X+Y \sim \Gamma(\alpha + c,\beta)$ e viceversa.
\end{enumerate}
\textbf{Dimostrazione}
\begin{enumerate}
\item $M_Y(t)= E(e^{tcX})=M_X(ct)=\frac{1}{(1-\beta ct)^\alpha} \qquad t<\frac{1}{c\beta}$
\item $$M_{X+Y}(t)=E(e^{t(X+Y)})=E(e^{tX}e^{tY})=E(e^{tX})E(e^{tY})$$
$$=M_X(t)M_Y(t)=\frac{1}{(1-\beta t)^\alpha}\times \frac{1}{(1-\beta ct)^c}=\frac{1}{(1-\beta ct)^\alpha +c}$$
\end{enumerate}
