\label{capitolo4}
\section{Verifica di ipotesi}
\subsection{Definizioni}
Un'\emph{ipotesi statistica} � un'asserzione o una congettura sulla fdr F.
Se F, e quindi la corrispondente densit� f, � nota a meno di un parametro $\theta=(\theta_1,\dots,\theta_m)\in\Theta\subset\mathbb{R}^m$, l'ipotesi statistica � un'asserzione su $\theta$.
Un'ipotesi � \emph{semplice} se specifica completamente la f.d.r. F della popolazione; � \emph{composta} se non � semplice.\\
In questo capitolo ci occuperemo di ipotesi che riguardano solo il parametro incognito. Nei problemi che affronteremo sono presenti due ipotesi chiamate $H_0$ e $H_1$ una � detta \emph{ipotesi nulla} mentre l'altra � detta \emph{ipotesi alternativa}.
Il vero valore che il parametro $\theta$ assume in natura � compatibile solo o con $H_0$ o con $H_1$ ma non con entrambi.\\
Dobbiamo ora stabilire una regola che ci permetta di decidere tra le due ipotesi. Decidiamo di partizionare l'insieme $\mathbb{R}^n$ di tutte le realizzazioni campionarie $X_1,\dots,X_n$ in due regioni $\mathcal{G}$ e $\mathcal{G}^c$ ed effettuiamo il campionamento; se $(x_1,\dots,x_n)\in \mathcal{G}$ rifiutiamo $H_0$ se invece $(x_1,\dots,x_n)\in \mathcal{G}^c$ allora accettiamo $H_0$.
La $\mathcal{G}$ � detta regione di \emph{rifiuto} o regione \emph{critica} mentre $\mathcal{G}^c$ � detta regione di \emph{accettazione}.\\
Pu� capitare di prendere delle decisioni sbagliate; gli errori che si possono commettere sono di due tipi:
\begin{description}
    \item[Errore di I tipo o prima specie:] quando rifiutiamo $H_0$ ma $H_0$ � vera.
	\item[Errore di II tipo o seconda specie:] quando accettiamo $H_0$ ma $H_0$ � falsa.
\end{description}
Questi due errori non possono essere calcolati con precisione ma possiamo calcolare la loro probabilit�. Sia $\alpha(\theta)$ la probabilit� di errore del \emph{I} tipo e sia $\beta(\theta)$ la probabilit� di errore di \emph{II} tipo allora:
\begin{equation}
	\alpha(\theta)=P_\theta("Rifiutare \ H_0")=P_\theta((x_1,\dots,x_n) \in \mathcal{G}), \qquad \theta\in\Theta_0
\end{equation}
\begin{equation}
	\beta(\theta)=P_\theta("Accettare \ H_0")=P_\theta((x_1,\dots,x_n) \in \mathcal{G}^c), \qquad \theta\in\Theta_1
\end{equation}
\subsection{Lemma di Neyuman-Pearson}
Un buon test � tale per cui $\alpha$ e $\beta$ sono trascurabili; ma in realt� esiste un trade-off cosicch� � impossibile minimizzare entrambi. Perci� si procede a fissare un valore per l'errore che si ritiene pi� grave (quello di I tipo) e poi si minimizza il secondo. Un test creato secondo questo criterio � detto \emph{test uniformemente pi� potente}.
Nel caso di ipotesi entrambi semplice il test creato � specificato dal \emph{Lemma di Neyman-Pearson}.\\
\\
Sia $X_1,\dots,X_n$ un campione casuale con verosimiglianza $L_\theta(x_1,\dots,x_n)$ e supponiamo di voler verificare $H_0:\theta=\theta_0$ contro $H_1:\theta=\theta_1$.
Sia $\mathcal{G}$ la regione critica definita da:
$$\mathcal{G}=\Bigg\{(x_1,\dots,x_n): \; \frac{L_{\theta_0}(x_1,\dots,x_n)}{L_{\theta_1}(x_1,\dots,x_n)}\leq \delta\Bigg\}$$
Allora $\mathcal{G}$ � la regione critica che genera massima potenza fra tutte le regioni critiche di ampiezza minore o uguale all'ampiezza di $\mathcal{G}$.\\
\textbf{Dimostrazione} Dobbiamo dimostrare che per qualunque regione critica $\mathcal{F}$ di ampiezza al pi� pari a quella di $\mathcal{G}$ cio� tale che $P_{\theta_0}(\mathcal{F})\leq P_{\theta_0}(\mathcal{G})$ vale $P_{\theta_1}(\mathcal{F})\leq P_{\theta_1}(\mathcal{G})$.
\\
Osserviamo che possiamo rappresentare $\mathcal{F}$ e $\mathcal{G}$ come:
$$\mathcal{F}=(\mathcal{F}\cap\mathcal{G})\cup(\mathcal{F}\cap\mathcal{G}^c),\qquad
\mathcal{G}=(\mathcal{F}\cap\mathcal{G})\cup(\mathcal{F}^c\cap\mathcal{G})$$
allora
\begin{equation}\label{np1}
	P_{\theta_i}(\mathcal{F})\leq P_{\theta_i}(\mathcal{G}) \ se \ solo \ se \ P_{\theta_i}(\mathcal{F}\cap\mathcal{G}^c)\leq P_{\theta_i}(\mathcal{F}^c\cap\mathcal{G}) \quad per \ i=0,1.
\end{equation}
Inoltre
$$L_{\theta_0}(x_1,\dots,x_n)\leq \delta L_{\theta_1}(x_1,\dots,x_n) \qquad \forall(x_1,\dots,x_n)\in\mathcal{G}$$
$$L_{\theta_1}(x_1,\dots,x_n) < L_{\theta_0}(x_1,\dots,x_n)/\delta  \qquad \forall(x_1,\dots,x_n)\in\mathcal{G}^c$$
da cui otteniamo che:
\begin{equation}\label{np2}
P_{\theta_0}(A)\leq \delta P_{\theta_1}(A)\quad \forall A \subset \mathcal{G}
\end{equation}
\begin{equation}\label{np3}
P_{\theta_1}(B)\leq \delta P_{\theta_0}(B)\quad \forall B \subset \mathcal{G}^c
\end{equation}
Infatti
$$P_{\theta_1}(B)=\int_B L_{\theta_1}(x_1,\dots,x_n) dx_1\dots dx_n= \frac{1}{\delta}\int_B L_{\theta_0}(x_1,\dots,x_n) dx_1\dots dx_n=\frac{P_{\theta_0}(B)}{\delta}$$
Pertanto
$$P_{\theta_1}(\mathcal{F}\cap \mathcal{G}^c) \leq P_{\theta_0}(\mathcal{F} \cap \mathcal{G}^c)/\delta \quad [per \ la \ \ref{np3}]$$
$$\leq P_{\theta_0}(\mathcal{F}^c \cap \mathcal{G})/\delta \quad [per \ la \ \ref{np1} \ con \ i=0]$$
$$\leq \delta P_{\theta_1}(\mathcal{F} \cap \mathcal{G}^c)/\delta \quad [per \ la \ \ref{np2}]$$
$$= P_{\theta_1}(\mathcal{F} \cap \mathcal{G}^c)$$
