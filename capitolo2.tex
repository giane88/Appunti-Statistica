\label{capitolo2}
\section{Intervalli di confidenza}
\subsection{Intervalli di confidenza per la media}
Abbiamo visto prima come nel caso di un campione casuale $X_1, \dots ,X_n$ estratto da una popolazione di densit� gaussiana di parametri $\mu , \sigma$ si possa stimare la media con la \emph{media campionaria} $\overline{X}$ e la varianza con la \emph{varianza campionaria} $S^2$.\\
Questo tipo di stima per� non ha molto senso in quanto sappiamo che:
$$P_{\mu,\sigma^2}(\overline{X}=c)=0$$
Ovvero � nulla la probabilit� che $\overline{X}$ assuma il vero valore di $\mu$.\\
Partiamo dal caso in cui solo la media $\mu$ � incognita mentre la varianza � nota e pari ad un valore fissato; noi possiamo allora stabilire una regione di probabilit� in cui possiamo stabilire con una certa precisione la probabilit� che il vero valore di $\mu$ sia in quella regione.
$$P_{\mu,\sigma^2}\Bigg(-\epsilon<\frac{\overline{X}-\mu}{\sigma/\sqrt{n}}<\epsilon\Bigg)=\gamma$$

Dove $\gamma$ � la confidenza con la quale siamo certi che il parametro stimato cada nell'intervallo.
Svolgendo i calcoli sopra per isolare $\mu$ troviamo che gli estremi dell'intevallo di confidenza sono:
$$\mu \in \Bigg( \overline{X}-z_{\frac{1+\gamma}{2}}\frac{\sigma}{\sqrt{n}} \ ; \ \overline{X}+z_{\frac{1+\gamma}{2}}\frac{\sigma}{\sqrt{n}}\Bigg)$$
\\
\\
Nel caso in cui la media non sia nota si pu� utilizzare la varianza campionaria per stimare un IC per la media in questo caso le formule diventano:
$$P_{\mu,\sigma^2}\Bigg(-\epsilon<\frac{\overline{X}-\mu}{S/\sqrt{n}}<\epsilon\Bigg)=\gamma$$
Questa volta la quantit� $\sqrt{n}(\overline{X}-\mu)/S$ � una t-student con n-1 gradi di libert� $t_{n-1}$ perci� l'iontervallo di confidenza diventa:
$$\mu \in \Bigg( \overline{X}-t_{n-1}\big(\frac{1+\gamma}{2}\big)\frac{\sigma}{\sqrt{n}} \ ;\ \overline{X}+t_{n-1}\big(\frac{1+\gamma}{2}\big)\frac{\sigma}{\sqrt{n}}\Bigg)$$
\subsection{Intervalli di confidenza per la varianza}
Nel caso in cui vogliamo trovare un intervallo di confidenza per la varianza quando la media � incognita, partiamo dalla quantit� aleatoria $S^2(n-1)/\sigma^2\sim \chi^2_{n-1}$ fissato un certo valore $\gamma$ dobbiamo fissare gli estremi dell'intervallo che come prima:
$$P_{\mu,\sigma^2}\Bigg(a<\frac{S^2(n-1)}{\sigma^2}<b\Bigg)=\gamma$$
Essendo la distribuzione $\chi^2$ una distribuzione asimmetrica possiamo ritrovarci in tre casi specifici:
\begin{enumerate}
\item $$\sigma^2 \in \bigg(\frac{S^2(n-1)}{\chi^2_{n-1}(\gamma)} \ ; \ +\infty \bigg)$$
\item $$\sigma^2 \in \bigg(0 \ ; \ \frac{S^2(n-1)}{\chi^2_{n-1}(1-\gamma)} \bigg)$$
\item $$\sigma^2 \in \bigg(\frac{S^2(n-1)}{\chi^2_{n-1}(\frac{1+\gamma}{2})} \ ; \ \frac{S^2(n-1)}{\chi^2_{n-1}(\frac{1-\gamma}{2})} \bigg)$$

\end{enumerate}
Nel caso di media nota i ragionamenti appena fatti sono ancora validi ma stimiamo $\sigma^2$ con la quantit�:
$$S^2_0= \frac{\sum^n_{j=1}(X_j-\mu)^2}{n}$$
